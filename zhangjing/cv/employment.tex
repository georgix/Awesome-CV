%-------------------------------------------------------------------------------
%	SECTION TITLE
%-------------------------------------------------------------------------------
\cvsection{Employment}


%-------------------------------------------------------------------------------
%	CONTENT
%-------------------------------------------------------------------------------
\begin{cventries}

%---------------------------------------------------------
\cventry
{Senior Android Software Engineer} % Job title
{Fusion Entertainment Inc.} % Organization
{Auckland, New Zealand} % Location
{Sept. 2016 - Now} % Date(s)
{
  \begin{cvsubentries} % Description(s) of tasks/responsibilities
    \cvsubentry
      {}
      {Hands Free Android App}
      {2016, 2018}
      {Developed an Android hands free app on two platforms, supporting making phone calls from headunit
      through mobile phone.}
    \cvsubentry
      {}
      {Linux Platform Bring Up}
      {2017-2018}
      {Worked on AM3358 bring up, developing keypad, rotator drivers, and integrating DDR(uboot), LCD(uboot and linux), graphics, splashscreen (uboot and linux) and Qt drivers and packages.}
    \cvsubentry
      {}
      {Multiple Audio Sources Handling}
      {2018}
      {With different audio sources on the system, a prioritized scheme was properly implemented.}
    \cvsubentry
      {}
      {Qt QML App Porting}
      {2019}
      {Create scalable QML app for different screen sizes.}
  \end{cvsubentries}
}

%---------------------------------------------------------
  \cventry
    {Senior Android Performance Engineer} % Job title
    {Lenovo Inc.} % Organization
    {Beijing, China} % Location
    {Apr. 2015 - Jul. 2016} % Date(s)
    {
      \begin{cvsubentries} % Description(s) of tasks/responsibilities
        \cvsubentry
          {}
          {System Performance Tuning}
          {2015-2016}
          {Optimize Android smart phone performance: Z1(based on Qualcomm 8974), Z2, Z2 pro(based on Qualcomm 8996).
           That covered power on/off time, wake up time, touch response time, benchmarking and etc.
           The Antutu test score was 145,741, defeating all known smart phones at that time.}
        \cvsubentry
          {}
          {FSFS}
          {2015-2016}
          {F2FS is a file system started by Samsung, specially designed for NAND device.
           AOSP had initial but incomplete support for the new file system.
           With a lot effort in linux kernel, AOSP build system, lk, fastboot, vold and factory mode,
           F2FS worked on Z2 smart phones, achieving more than 100\% improvement in random write throughput.}
        \cvsubentry
          {}
          {CPU Overclock}
          {2016}
          {MSM8996 cpu frequency was overclocked from 2.15GHz to 2.3GHz, gaining more than 5\% performance improvement.
           This was a marketting highlight of ZUK Z2.}
        \cvsubentry
          {}
          {Issue Analysis}
          {2015-2016}
          {Stability, Memory, Battery, Overheating issue analysis.}
      \end{cvsubentries}
    }

%---------------------------------------------------------
  \cventry
    {Senior Android Software Engineer/Techical Deputy} % Job title
    {Mediatek Inc.} % Organization
    {Beijing, China} % Location
    {Jun. 2011 - Apr. 2015} % Date(s)
    {
      \begin{cvsubentries} % Description(s) of tasks/responsibilities
        \cvsubentry
          {}
          {Android CTS}
          {2011-2014}
          {Android Compatibility Test Suite(CTS) is a mandatory test from google for all devices to market as a Android compatible device. There are more than 20,000 cases covering almost every aspect of Android AOSP requirement. My Responsibility was to insure all Meditek chipsets can pass CTS certification and assist customers to pass CTS. I had to find the root cause of issue blocking the running of CTS test, and sometimes collaberate with google's Technical Account Manager(TAM) to feedback and ask for waive for some flaw disigned testcases. With some modification, I worked out a windows version CTS test runner which was not supported by AOSP, and consequently all RD and QA could run CTS test on their PC even most of them didn't have a Linux environment.}
        \cvsubentry
          {}
          {Permission \& Security}
          {2012-2014}
          {Most Android engineers in MTK were transfered from MTK feature phone team, with little system security concept in mind. Security problem became fatal when we seek collaberation with tier one mobile phone manufactures and I was assigned to solve this problem. After discussion with key customers, I worked out a permission control guideline on root privilages, group permission, global writable file. Then I went on to coordinate developing resources across teams from modem, bluetooth, wifi, gps, fm, factory tool, debug tool, OTA, video, audio, graphics, sdcard, and finally we were recognized for our quick response and excellent work by the customers. This was a big step for Mediatek's success on Android smart phone chipset market.}
        \cvsubentry
          {}
          {Android Exception Engine}
          {2012-2015}
          {Android Exception Engine(AEE) a Mediatek proprietary Android device side runtime debug tool, developed based on AOSP debuggerd, with a lot of critical debug function improvement, including runtime exception snapshot, runtime call stack dump, coredump, crash dump and etc. It has cross layer full support of lk, linux kernel, native c/c++, and Java, and works as the Mediatek Android debug core system. This was a project maintained by 5-6 members, and my responsibility mostly went to lk and linux kernel, and sometimes native c/c++.}
        \cvsubentry
          {}
          {Mediatek GAT}
          {2011-2012}
          {Graphical Android Toolkit is a Mediatek proprietary PC side debug tool built on top of google's DDMS, with many new features, including native(non-java) process debug, kernel/native c/java runtime call stack dump, kernel log capturing, ftrace support, interaction with Mediatek's device side AEE tool and etc. I was one of two main contributors in this new startup project, focusing on the Android related logic implementation.}
        \cvsubentry
          {}
          {Mediatek Logviewer}
          {2011-2014}
          {Mediatek Logviewer is another Mediatek proprietary PC side debug tool, developed with RCP technology. It is designed to work in offline mode(without connection to Android device required) to facilitate the analysis of logs saved by GAT and AEE. Also in later phase with some kind of intelligence to dispatch the bug to the right module owner which is quite helpful to QA team and greatly boost the efficiecy of problem solving. Designed with python plugin support from the start, this allows great flexibility to improve the intelligence to classify bugs with the help of experts from different domain. I was one of two main contributors in this project and was responsible for the initial plugins support for Android AOSP java/native/kernel layer exception analysis}
        \cvsubentry
          {}
          {AOSP Toolchain}
          {2011-2014}
          {In the begining time of AOSP, google did not manage to catch up the pace of upstream toolchain, like gcc, gdb, python. In order to use python extensions for gdb, we had to maintain private toolchain. I started from latest upstream gdb and python, and with a lot search and survey, finally got it work in the right way. We had a working gdb with python extension support on both linux and windows platform.}
        \cvsubentry
          {}
          {Stability Issue Analysis}
          {2013-2015}
          {At the product develop stage, there is active feature change, and this usually leads to system unstability. One of my key responsibility was to analyse the root cause of kernel crash and watchdog reset, boot failure, device freeze and etc. I got comprehensive understanding to many kernel modules through long-term practice, including phisical and virtual memory management, file system, thread, task, scheduler, lock. Later my experience in analysing these failures were turned into the python scripts in Mediatek Logviewer, resulting to a great relief of my work presure and faster problems solving.}
        \cvsubentry
          {}
          {Linux Kernel Mini-Dump}
          {2014-2015}
          {There are two major options to debug linux kernel crash, one is the oops message dump to kernel log, the other is full ram dump. Oops message only contains very basic call stack information which is hardly enough to analyse the crash, and ram dump is very helpful but with very huge size. And with the expansion of Android smart phone DRAM size, the situation was getting worse. Then I designed a new solution called mini-dump, with size only 1-3MB, could give enough debug information in 70\% cases . Mini-dump is a ELF format kernel dump, can be directly loaded by gdb, result in much faster debugging.}
        \cvsubentry
          {}
          {AOSP Develop \& Debug Support}
          {2013-2015}
          {With my experience increasing during practice, I gave suggestions to other team especially on permission, inter-module communication, and debugging. At the same time, our bug analysis system became more powerful with the contribution of different teams.}
      \end{cvsubentries}
    }

%---------------------------------------------------------
\end{cventries}
